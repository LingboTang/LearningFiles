%%%%%%%%%%%%%%%%%%%%%%%%%%%%%%%%%%%%%%%%%%%%%%%%%%%%%%%%%%%%%%%%%%%%%%%%%%%
%
% Template for a LaTex article in English.
%
%%%%%%%%%%%%%%%%%%%%%%%%%%%%%%%%%%%%%%%%%%%%%%%%%%%%%%%%%%%%%%%%%%%%%%%%%%%

\documentclass{article}

% AMS packages:
\usepackage{amsmath, amsthm, amsfonts}

% Theorems
%-----------------------------------------------------------------
\newtheorem{thm}{Theorem}[section]
\newtheorem{cor}[thm]{Corollary}
\newtheorem{lem}[thm]{Lemma}
\newtheorem{prop}[thm]{Proposition}
\theoremstyle{definition}
\newtheorem{defn}[thm]{Definition}
\theoremstyle{remark}
\newtheorem{rem}[thm]{Remark}

% Shortcuts.
% One can define new commands to shorten frequently used
% constructions. As an example, this defines the R and Z used
% for the real and integer numbers.
%-----------------------------------------------------------------
\def\RR{\mathbb{R}}
\def\ZZ{\mathbb{Z}}

% Similarly, one can define commands that take arguments. In this
% example we define a command for the absolute value.
% -----------------------------------------------------------------
\newcommand{\abs}[1]{\left\vert#1\right\vert}

% Operators
% New operators must defined as such to have them typeset
% correctly. As an example we define the Jacobian:
% -----------------------------------------------------------------
\DeclareMathOperator{\Jac}{Jac}

%-----------------------------------------------------------------
\title{Implementing Jarzynski Equality in Python}
\author{Lingbo Tang\\
  \small Dept. Computing Science\\
  \small University of Alberta\\
  \small Canada
}

\begin{document}
\maketitle

\abstract{Jarsynski Equality is a neat equality that reveals that the system Free Energy could be calculated even when the system is not in equilibrium state.}

\section{Introduction}

In general, the Jarzynski Equality could look like this:

\begin{equation}\label{eq:general}
  \Delta F = F(\lambda_{t}) - F(\lambda_{0}) <= < W >
\end{equation}

and the integration form looks like:

\begin{equation}\label{eq:integration}
  W_{0\rightarrow t} = \int_{0}^{t} dt^{'} \frac{\partial \lambda_t^{'}}{\partial t^{'}} [\frac{\partial \tilde{H} (r, p; \lambda)}{\partial \lambda} ]_{(r,p; \lambda)} = (r_{t^{'}}, p_{t^{'}} ; \lambda_{t^{'}})
\end{equation}

One can refer to equations like this: see equation (\ref{eq:general}). One can also
refer to sections in the same way: see section \ref{sec:nothing}. Or
to the bibliography like this: \cite{Cd94}.

\subsection{Subsection}\label{sec:nothing}

More text.

\subsubsection{Subsubsection}\label{sec:nothing2}

More text.

% Bibliography
%-----------------------------------------------------------------
\begin{thebibliography}{99}

\bibitem{Cd94} Author, \emph{Title}, Journal/Editor, (year)

\end{thebibliography}

\end{document}
